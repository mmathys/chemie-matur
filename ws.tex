%% ------------------------------------------------------------------------------------------------ %
% ZUSAMMENFASSUNG
% WAHRSCHEINLICHKEIT UND STATISTIK
% ------------------------------------------------------------------------------------------------ %
% Autor: J�r�me Dohrau
% Email: dohrauj @ ...
% Die Zusammenfassung darf gerne f�r den eigenen Gebrauch angepasst werden. Falls jemand einen
% Fehler findet, w�rde ich mich �ber eine Benachrichtigung per Email freuen.
% ------------------------------------------------------------------------------------------------ %


\documentclass[a4paper,twocolumn]{article}


% ------------------------------------------------------------------------------------------------ %
% ALLGEMEINE PAKETE
% ------------------------------------------------------------------------------------------------ %


% Silbentrennung, Sonderzeichen ect.
\usepackage[german]{babel}
\usepackage[T1]{fontenc}
\usepackage[utf8]{inputenc}


% Mathematische Zeichen
\usepackage{amssymb}
% Lehrs�tze
\usepackage{amsthm}
% Mathematische Extras (Matrizen ect.)
\usepackage{amsxtra}

% F�r kompakte Listen
\usepackage{paralist}

% Rahmen
\usepackage{framed}

% Grafik
%\usepackage{graphicx}
\usepackage{ifpdf}
\usepackage{psfrag}
\usepackage{epstopdf}

\usepackage{titlesec}
\usepackage{titletoc}

% Graphs and Automata
\usepackage{gastex}

% farben
\usepackage{color}
\usepackage{bbold}




% ------------------------------------------------------------------------------------------------ %
% FARBEN
% ------------------------------------------------------------------------------------------------ %


\definecolor{headtext}{rgb}{0.50,0.50,0.50}
\definecolor{foottext}{rgb}{0.50,0.50,0.50}
\definecolor{headsepline}{rgb}{0.80,0.80,0.80}
\definecolor{footsepline}{rgb}{0.80,0.80,0.80}


% Schattierung f�r Hinweisboxen
\definecolor{shadecolor}{rgb}{0.88,0.91,0.95}

\definecolor{highlightcolor}{rgb}{1.00,0.50,0.50}


% ------------------------------------------------------------------------------------------------ %
% SEITEN- UND TEXTLAYOUT
% ------------------------------------------------------------------------------------------------ %


% Seitenr�nder
\usepackage[left=10mm,right=10mm,top=20mm,bottom=20mm]{geometry}

% Zeileneinzug am Anfang eines Absatzes
\setlength{\parindent}{0em}


% ------------------------------------------------------------------------------------------------ %
% KOPF- UND FUSSZEILEN
% ------------------------------------------------------------------------------------------------ %


\usepackage{fancyhdr}
\pagestyle{fancy}

\fancyhead{} % l�schen
\fancyhead[L]{\color{headtext}\uppercase{Chemie Matur Zusammenfassung}}
\fancyhead[R]{\color{headtext}\leftmark}

\fancyfoot{} % l�schen
\fancyfoot[L]{\color{foottext}\uppercase{Seite} \thepage}
\fancyfoot[R]{\color{foottext}\uppercase{Max Mathys}}

\renewcommand{\headrulewidth}{1pt}
\renewcommand{\headrule}{{\color{headsepline}\hrule width\headwidth height\headrulewidth \vskip-\headrulewidth}}
\renewcommand{\footrulewidth}{1pt}
\renewcommand{\footrule}{{\color{footsepline}\vskip-\footruleskip\vskip-\footrulewidth\hrule width\headwidth height\footrulewidth\vskip\footruleskip}}


% ------------------------------------------------------------------------------------------------ %
% LEHRS�TZE (DEFINITIONEN ETC.)
% ------------------------------------------------------------------------------------------------ %


\newtheoremstyle{defstyle}
	% Abstand oben
	{10pt}
	% Abstand unten
	{10pt}
	% Schriftart
	{\normalfont}
	% Zeileneinzug (leer = kein Zeileneinzug, \parindent = Absatz-Zeileneinzug)
	{}
	% Titel Schriftart
	{\normalfont\bfseries}
	% Zeichen nach dem Titel
	{:} 
	% Leerraum nach dem Titel
	{ }
	% Lehrsatzkopf definieren (leer = normal)
	{#3}

\newtheoremstyle{thmstyle}
	% Abstand oben
	{10pt}
	% Abstand unten
	{10pt}
	% Schriftart
	{\normalfont}
	% Zeileneinzug (leer = kein Zeileneinzug, \parindent = Absatz-Zeileneinzug)
	{}
	% Titel Schriftart
	{\normalfont\bfseries}
	% Zeichen nach dem Titel
	{:} 
	% Leerraum nach dem Titel
	{ }
	% Lehrsatzkopf definieren (leer = normal)
	{\thmname{#1}\thmnumber{ #2}\thmnote{ (#3)}}

\theoremstyle{defstyle}
% Definition
\newtheorem*{definition}{Definition}

\theoremstyle{thmstyle}
% 
\newtheorem*{theorem}{Satz}
% Beachte
\newtheorem*{tnote}{Hinweis}

% Beispiel
\newtheorem*{example}{Beispiel}

\newenvironment{note}
	{\begin{snugshade}\begin{tnote}}
	{\end{tnote}\end{snugshade}}
	
\newenvironment{highlight}
	{\begin{snugshade}}
	{\end{snugshade}}


% ------------------------------------------------------------------------------------------------ %
% ABBILDUNGEN
% ------------------------------------------------------------------------------------------------ %


\usepackage[bf]{caption}

\addto\captionsgerman{
    \renewcommand{\figurename}{Abb.}
    \renewcommand{\tablename}{Tab.}}

% Abbildungs�berschrift
\renewcommand{\captionfont}{\footnotesize\sffamily}


% ------------------------------------------------------------------------------------------------ %
% SELBSTDEFINIERTE BEFEHLE
% ------------------------------------------------------------------------------------------------ %


\newcommand{\todo}[1]{{\definecolor{shadecolor}{rgb}{1.00,0.30,0.30}\begin{snugshade}{\bf TODO:} #1\end{snugshade}}}

\newcommand{\R}{\mathbb{R}}
\newcommand{\N}{\mathbb{N}}
\renewcommand{\P}{\mathbb{P}}
\newcommand{\E}{\mathbb{E}}

\renewcommand{\d}{\mathrm{d}}

\renewcommand{\theta}{\vartheta}
\renewcommand{\phi}{\varphi}

\newcommand{\var}{\mathrm{Var}}
\newcommand{\cov}{\mathrm{Cov}}
\newcommand{\sd}{\mathrm{sd}}
\newcommand{\corr}{\mathrm{Corr}}

\newcommand{\abs}[1]{\lvert #1 \rvert}


% ------------------------------------------------------------------------------------------------ %
% LAYOUT INHALTSVERZEICHNIS UNDSO
% ------------------------------------------------------------------------------------------------ %


\renewcommand{\thepart}{\Alph{part}}
\renewcommand{\thesection}{\Alph{part}.\arabic{section}}

\titlecontents{part}[0em]%
{\vspace{1em}\bf\large}{\thecontentslabel\enskip}{}%
{\hfill\contentspage}

\titlecontents{section}[0em]%
{\vspace{0.3em}\bf}{\thecontentslabel\enskip}{}%
{\hfill\contentspage}

\titlecontents{subsection}[1em]%
{}{\thecontentslabel\enskip}{}%
{\titlerule*[0.6em]{.}\contentspage}

\titlecontents{subsubsection}[2em]%
{}{\thecontentslabel\enskip}{}%
{\titlerule*[0.6em]{.}\contentspage}


% ------------------------------------------------------------------------------------------------ %
% INHALT
% ------------------------------------------------------------------------------------------------ %


\begin{document}

\setcounter{tocdepth}{2}
\tableofcontents

\hfill\newpage

\setcounter{part}{1}
\setcounter{section}{0}
\part*{Atomlehre}
\addcontentsline{toc}{part}{Atomlehre}

\section{Atommodelle}

\paragraph{Dalton.}

Im Dalton-Modell stellt man sich die Atome als Kugeln vor. Nach Ansicht von Dalton besteht jedes Element aus gleichen kleinsten Teilchen, welche auch er als Atome bezeichnet.

\paragraph{Rutherford.}

Kern-Hülle-Modell; ein Atom hat einen positiv geladenen Kern. Diese positiven Anteile bekamen den Namen Protonen. Um den Kern herum kreisen Elektronen auf Kreisbahnen und stellen den negativ geladenen Teil des Atoms dar. Erscheint ein Atom nach außen hin elektrisch neutral, muss der Anteil an positiven und negativen Ladungen gleich groß sein.

\paragraph{Bohr.}

Elektronen können nur ganz bestimmte Energiezustände einnehmen. Elektronen können allerdings nur ganz bestimmte - also nicht beliebige - Abstände vom Kern einnehmen. Diese jeweiligen stabilen Kreisbahnen verhindern den Sturz der Elektronen auf den Atomkern.

\begin{itemize}
	\item K-Schale:  2 Elektronen
	\item L-Schale:  8 Elektronen
	\item M-Schale: 18 Elektronen
	\item N-Schale: 32 Elektronen
\end{itemize}

\begin{definition}[Valenzelektronen]
	Elektronen auf nicht gesättigten Elektronenschalen
\end{definition}

\setcounter{part}{2}
\setcounter{section}{0}
\part*{Bindungslehre}
\addcontentsline{toc}{part}{Bindungslehre}

\section{Kovalente Bindung}

\subsection{Strukturschreibweisen}

\begin{itemize}
	\item Strukturformel
	\item Skelettformel = Lewis-Formel = Strich-Formel
	\item Gruppenformel
\end{itemize}

\subsection{Struktur und Geometrie von Molekülen, Elektronennegativität, Polarität}



\subsubsection{Orbitale}

\begin{itemize}
	\item s: 2 Elektronen
	\item p: 6 Elektronen
	\item d: 10 Elektronen
	\item f: 14 Elektronen
\end{itemize}

\subsubsection{Elektronenkonfiguration Schreibweise}

C: $1s^2 \ 2s^2 \ 2p^2$ oder $[He] \ 2s^2 \ 2p^2$

\subsubsection{Elektronennegativität}

\begin{definition}[Elektronennegativität]
	$\chi$; relatives Mass für die Fähigkeit eines Atoms, in einer chemischen Bindung Elektronenpaare an sich zu ziehen. Von 0.7 bis 4.
\end{definition}

Bei $\Delta_\chi \geqslant 1.8$: Ionenbindung.

\begin{definition}[Bindungspolarität]
	Eine Bindung eines Atoms mit hoher und eines Atoms mit tiefer Elektronegativität ist
	polar. Auf der Seite des Atoms mit der höheren Elektronegativität ist die Partialladung
	negativ ($\delta -$), auf der Seite des Atoms mit der tieferen Elektronegativität positiv ($\delta +$).
	Beispiel: Chlor (Cl) hat eine Elektronegativität von 3.0, bei Brom (Br) beträgt sie nur
	2.8. Gehen nun ein Chlor- und ein Brom-Atom eine Verbindung ein, ist diese polar (Cl :
	$\delta -$, Br : $\delta +$). Da die Differenz nur gerade 0.2 beträgt, ist die Bindung nur schwach polar.
	Je höher die Differenz der beiden Elektronegativitäten, desto stärker ist die Bindungspolarität.
\end{definition}

\subsection{Zwischenmolekulare Kräfte}

\paragraph{Wasserstoffbrücken.} 

Wasserstoffbrücken sind elektrostatische Kräfte zwischen Wasserstoffatomen, die an F-,
O- oder N-Atome gebunden sind und den freien Elektronenpaaren solcher Atome in benachbarten
Molekülen. Sie wirken, weil Wasserstoff von allen Nichtmetallen die kleinste
Elektronegativität hat.

\paragraph{Dipol-Dipol-Wechselwirkung.}

$\Delta_\chi \geqslant 0.5$. Zwischen Dipol-Molekülen wirken die Dipol-Dipol-Kräfte. Diese Kräfte sind relativ stark
und von der Molekülgestalt und der Bindungspolarität (Differenz der Elektronegativität)
abhängig.

\paragraph{Van der Waals-Kräfte.}

 Die Van-der-Waals-Kräfte entstehen aufgrund der zeitweise asymmetrischen Ladungsverteilung,
 die bei der Bewegung von Elektronen um einen Atomkern auftreten. Es entstehen
 momentane Dipole. Je grösser die Molekülmasse und die Moleküloberfläche, desto stärker
 die Van-der-Waals-Kräfte. Bei kleinen Dipol-Molekülen sind die Dipol-Dipol-Kräfte
 in der Regel stärker als die Van-der-Waals-Kräfte.


\subsection{Eigenschaften molekularer Stoffe}

\paragraph{Schmelzpunkt, Siedepunkt.}

Eher tief, viele Molekülverbindungen sind bei Raumtemperatur flüssig oder gasförmig. Abhängig von zwischenmolekularen Kräften (ZMK).

\paragraph{Löslichkeit.}

Abhängig von ZMK. Polare Moleküle wasserlöslich, unpolare löslich in unpolaren Lösungsmitteln (Benzin).

\paragraph{Sonstiges.}

Elektrische Nichtleiter.



%http://paedubucher.ch/passerelle/chemie/heft_105.pdf

\section{Ionenbindung}

\subsection{Struktur und Aufbau von Salzen}

Ionengitter mit Kationen und Anionen.

\begin{definition}[Kation]
	Gibt ein Metall-Atom Valenzelektronen ab, wird aus ihm ein positiv geladenes Metall-Ion, Kation.
\end{definition}

\begin{definition}[Anion]
	Nimmt ein Nichtmetall-Atom Valenzelektronen auf, wird aus ihm ein negativ geladenes Nichtmetall-Ion, Anion.
\end{definition}

\begin{definition}[Anorganische Salze]
	Es liegen Kationen von Metallen vor, die Anionen sind Nichtmetalle oder deren Oxide. Natriumchlorid: Na: Metall. Cl: Nichtmetall.
\end{definition}

\begin{definition}[Organische Salze]
	Mindestens ein Kation oder Anion ist eine organische Verbindung.
\end{definition}

\subsection{Entstehung von Salzen}

\begin{itemize}
	 \item Zunächst muss die Aktivierungsenergie zugeführt werden, damit die Reaktion in
	 Gang kommt. Natrium muss vom festen in den gasförmigen Zustand überführt werden,
	 dazu muss Sublimationsenergie aufgewendet werden. Da Chlor in der Natur nur
	 als Chlor-Verbindung $Cl_2$ vorkommt, müssen die beiden Chlor-Atome voneinander
	 getrennt werden. Dazu wird Bindungsenergie aufgewendet.
	 \item Die Natrium-Atome geben alle ihre Valenzelektronen ab. Die Natrium-Kationen
	 ($N a+$) verlieren damit ihre Valenzschale und haben nun nur noch 2 Elektronenschalen
	 (Elektronenkonfiguration: 2/8). Die Entfernung der Valenzelektronen erfordert
	 Ionisierungsenergie. Bei einer Reaktion mit Nichtmetallen zu Salz geben
	 Hauptgruppen-Metalle in der Regel alle ihre Valenzelektronen ab und erreichen dadurch
	 Edelgaskonfiguration.
	 \item Die Chlor-Atome füllen ihre Valenzschale mit den freigewordenen Elektronen. Dabei
	 wird der Energiebetrag der Elektronenaffinität freigesetzt. Aus den Chlor-Atomen
	 sind nun Chlor-Anionen ($Cl-$) geworden (Elektronenkonfiguration: 2/8/8, Edelgaskonfiguration).
	 \item $N a+$ und $Cl-$ verbinden sich zu einem Kristall (Ionengitter). Dabei wird die Gitterenergie
	 freigesetzt. Eine solche Verbindung wird als Ionenverbindung bezeichnet.
\end{itemize}

\subsection{Eigenschaften von Salzen}

\paragraph{Schmelzpunkt, Siedepunkt.}

Hoch, bei Raumtemperatur sind alle Salze fest Abhängig von Gitterenergie

\paragraph{Löslichkeit.}

Mehr oder weniger in Wasser löslich, je nach Gitterenergie und Hydratationsenergie. Faustregel: unlöslich, wenn beide Ionen Ladung 2 oder höher haben (Ausnahmen: AgCl, AgI) Nicht löslich in unpolaren Lösungsmitteln

\paragraph{Sonstiges.}

Spröde, in Lösung oder als Schmelze: elektrisch leitfähig (es können sich die geladenen Ionen als Ladungsträger frei in der Flüssigkeit
bewegen). Feststoff: nicht leitend.

\paragraph{Reaktionen.}

Salzbildung, Elektrolyse, Lösen, Fällen
%http://paedubucher.ch/passerelle/chemie/heft_105.pdf

\section{Metallbindung}

\subsection{Aufbau von Metallen}

Im festen Zustand bilden Metall-Atome ein Metallgitter. Die Atomrümpfe sind dicht gepackt.
Dazwischen bewegen sich die VE frei umher (Elektronengas). Zwischen den negativ
geladenen Elektronen und den positiv geladenen Atomrümpfen herrschen elektrostatische
Kräfte, welche das Gitter zusammenhalten (metallische Bindung).

\subsubsection{Bedeutung des Aufbaus}
\begin{itemize}
	\item Die verschiebbaren Elektronen ermöglichen die elektrische Leitfähigkeit.
	\item Die metallische Bindung (die starken Gitterkräfte) führen zu einer hohen Härte und
	hohen Schmelz- und Siedetemperaturen.
	\item Die Gitterebenen lassen sich leicht gegeneinander verschieben, wodurch Metalle
	duktil (verformbar) werden.
\end{itemize}

\subsection{Eigenschaften von Metallen}

\paragraph{Schmelzpunkt, Siedepunkt.}

In der Regel relativ hoch, abhängig von Anzahl Valenzelektronen, bei Raumtemperatur, fest (ausser Hg).

\paragraph{Löslichkeit.}

Unlöslich.

\paragraph{Sonstiges.}

Gute elektrische Leiter, gute Wärmeleiter, Metallglanz, duktil (verformbar).

\setcounter{part}{18}
\setcounter{section}{0}
\part*{Reaktionslehre}
\addcontentsline{toc}{part}{Reaktionslehre}

\section{Chemisches Rechnen}

\subsection{Stöchimetrisches Rechnen}

\begin{definition}[mol]
	$1 \ mol = 6.022 \cdot 10^{23}$ Teilchen
\end{definition}

\begin{definition}[n]
	Teilchenanzahl; in mol.
\end{definition}

\begin{definition}[m]
	Gewicht absolut; in g.
\end{definition}

\begin{definition}[M]
	Gewicht pro mol; in $g/mol$
\end{definition}

\begin{definition}[V]

	Volumen; in	\LARGE $\int_{- \infty}^{420} (weed)$
\end{definition}

\begin{definition}[c]
	Konzentration; in $mol/l$
\end{definition}

\subsection{Konzentrationsberechnungen}

\large{
	$n=\frac{m}{M}$
	\\ \\
	$c=\frac{n}{V}$
	\\ \\
	$n = \frac{V}{V_m}$
}


%TODO Konzentration
\section{Kinetik}

\subsection{Grundlagen}

TODO
\section{Chemisches Gleichgewicht}

\subsection{Grundlagen der Thermodynamik}

\large{

$ p \cdot v = n \cdot R \cdot T $

$R = 8.31448 \ \frac{J}{K \cdot mol}$
}

\subsection{Massenwirkungsgesetz}

\subsection{Die Gibbs-Energie}
\section{Ozon}
\section{Säure-Base Reaktionen}

\subsection{Definition nach Brönsted}

\subsection{Säure-Base Reaktionen (Protolyse)}

\subsection{pH Berechnungen}

\subsection{Neutralisationen}

\subsection{Titrationen}

\subsection{Puffer}

%http://paedubucher.ch/passerelle/chemie/heft_107.pdf
\section{Redox-Reaktionen}

\subsection{Galvanische Elemente}

\subsubsection{Batterien}

\subsubsection{Brennstoffzelle}

\subsubsection{Akkus}

\subsection{Elektrolyse}

%http://paedubucher.ch/passerelle/chemie/heft_108.pdf

\setcounter{part}{15}
\setcounter{section}{0}
\part*{Organische Chemie}
\addcontentsline{toc}{part}{Organische Chemie}

\section{Kohlenwasserstoffe}

\subsection{Alkane, Alkene, Alkine, Aromaten}

\subsubsection{Allgemein}

\subsubsection{Wichtige Vertreter}

\subsection{Nomenklatur}

\subsubsection{Allgemein}

\subsubsection{Funktionelle Gruppen}

\section{Funktionelle Gruppen}
\section{Erdöl}
\section{Kunststoffe}
\section{Aminosäuren und Proteine}


{\Large Mit <3 gemacht von Max Mathys}

\vspace{.7em}

\textbf{Schule}: MNG Rämibühl
\vspace{.2em}

\textbf{Jahr}: 2016
\vspace{.2em}

\textbf{Lehrerin}: Antoniadis

\label{lastpage}

\end{document}


% ------------------------------------------------------------------------------------------------ %