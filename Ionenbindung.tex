%http://paedubucher.ch/passerelle/chemie/heft_105.pdf

\section{Ionenbindung}

\subsection{Struktur und Aufbau von Salzen}

Ionengitter mit Kationen und Anionen.

\begin{definition}[Kation]
	Gibt ein Metall-Atom Valenzelektronen ab, wird aus ihm ein positiv geladenes Metall-Ion, Kation.
\end{definition}

\begin{definition}[Anion]
	Nimmt ein Nichtmetall-Atom Valenzelektronen auf, wird aus ihm ein negativ geladenes Nichtmetall-Ion, Anion.
\end{definition}

\begin{definition}[Anorganische Salze]
	Es liegen Kationen von Metallen vor, die Anionen sind Nichtmetalle oder deren Oxide. Natriumchlorid: Na: Metall. Cl: Nichtmetall.
\end{definition}

\begin{definition}[Organische Salze]
	Mindestens ein Kation oder Anion ist eine organische Verbindung.
\end{definition}

\subsection{Entstehung von Salzen}

\begin{itemize}
	 \item Zunächst muss die Aktivierungsenergie zugeführt werden, damit die Reaktion in
	 Gang kommt. Natrium muss vom festen in den gasförmigen Zustand überführt werden,
	 dazu muss Sublimationsenergie aufgewendet werden. Da Chlor in der Natur nur
	 als Chlor-Verbindung $Cl_2$ vorkommt, müssen die beiden Chlor-Atome voneinander
	 getrennt werden. Dazu wird Bindungsenergie aufgewendet.
	 \item Die Natrium-Atome geben alle ihre Valenzelektronen ab. Die Natrium-Kationen
	 ($N a+$) verlieren damit ihre Valenzschale und haben nun nur noch 2 Elektronenschalen
	 (Elektronenkonfiguration: 2/8). Die Entfernung der Valenzelektronen erfordert
	 Ionisierungsenergie. Bei einer Reaktion mit Nichtmetallen zu Salz geben
	 Hauptgruppen-Metalle in der Regel alle ihre Valenzelektronen ab und erreichen dadurch
	 Edelgaskonfiguration.
	 \item Die Chlor-Atome füllen ihre Valenzschale mit den freigewordenen Elektronen. Dabei
	 wird der Energiebetrag der Elektronenaffinität freigesetzt. Aus den Chlor-Atomen
	 sind nun Chlor-Anionen ($Cl-$) geworden (Elektronenkonfiguration: 2/8/8, Edelgaskonfiguration).
	 \item $N a+$ und $Cl-$ verbinden sich zu einem Kristall (Ionengitter). Dabei wird die Gitterenergie
	 freigesetzt. Eine solche Verbindung wird als Ionenverbindung bezeichnet.
\end{itemize}

\subsection{Eigenschaften von Salzen}

\paragraph{Schmelzpunkt, Siedepunkt.}

Hoch, bei Raumtemperatur sind alle Salze fest Abhängig von Gitterenergie

\paragraph{Löslichkeit.}

Mehr oder weniger in Wasser löslich, je nach Gitterenergie und Hydratationsenergie. Faustregel: unlöslich, wenn beide Ionen Ladung 2 oder höher haben (Ausnahmen: AgCl, AgI) Nicht löslich in unpolaren Lösungsmitteln

\paragraph{Sonstiges.}

Spröde, in Lösung oder als Schmelze: elektrisch leitfähig (es können sich die geladenen Ionen als Ladungsträger frei in der Flüssigkeit
bewegen). Feststoff: nicht leitend.

\paragraph{Reaktionen.}

Salzbildung, Elektrolyse, Lösen, Fällen